\documentclass[unicode,12pt]{beamer}% 'unicode'が必要
\usepackage{luatexja}% 日本語したい
\usepackage[ipaex]{luatexja-preset}% IPAexフォントしたい
\renewcommand{\kanjifamilydefault}{\gtdefault}% 既定をゴシック体に

\usepackage[author={narration}]{pdfcomment}
\definestyle{note}{icon=Insert,color=red} 
\newcommand{\pdfnarration}[1]{%
\onslide*<\value{beamerpauses}>{\pdfmargincomment[style=note,author=narration]{#1}}%
}

\title{日本語Beamer動作確認}
\author{tmiya}
\date{\today}

\begin{document}

\begin{frame}
\titlepage
\end{frame}

\begin{frame}{日本語がちゃんと表示されるか確認}
    こんにちは!このスライドは日本語です。

    \alert{赤字の部分もちゃんと見えてますか?}

    次のページでは長い文章でVOICEVOXの読み上げテストをします。
\pdfnarration{
こんにちは!このスライドは日本語です。
次のページでは長い文章でVOICEVOXの読み上げテストをします。
}
\end{frame}

\begin{frame}{VOICEVOX読み上げテスト用文章}
    みなさん、こんにちは!今日はBeamerで作ったスライドに自動で音声を付ける方法を紹介します。

    まず、このtexファイルをlualatexでコンパイルすると、注釈付きのPDFファイルが生成されます。
    ついで jbeam2vid スクリプトが PDF から注釈を抽出し、VOICEVOX を使って日本語音声を生成します。
    そして最後に PDF と音声ファイルを組み合わせて動画が生成されます。

    VOICEVOXを使って「ずんだもん」や「四国めたん」の声で音声を生成できますし、
    日本語の句読点・カタカナ・漢字もすべて完璧に読み上げられます!
    例:「こんにちは、世界!」「ぱちぱちぱち」「わーい!」
\pdfnarration{
みなさん、こんにちは!今日はBeamerで作ったスライドに自動で音声を付ける方法を紹介します。
まず、このtexファイルをlualatexでコンパイルすると、注釈付きのPDFファイルが生成されます。
ついで jbeam2vid スクリプトが PDF から注釈を抽出し、VOICEVOX を使って日本語音声を生成します。
そして最後に PDF と音声ファイルを組み合わせて動画が生成されます。
VOICEVOXを使って「ずんだもん」や「四国めたん」の声で音声を生成できますし、
日本語の句読点・カタカナ・漢字もすべて完璧に読み上げられます!
例:「こんにちは、世界!」「ぱちぱちぱち」「わーい!」}
\end{frame}

\begin{frame}{最後}
    \Huge ご清聴ありがとうございました!
\pdfnarration{
ご清聴ありがとうございました!
}
\end{frame}

\end{document}
