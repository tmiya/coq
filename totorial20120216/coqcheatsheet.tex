\documentclass{jarticle}
\usepackage{verbatim}
\usepackage{ascmac}
\usepackage{eclbkbox}
\usepackage{bussproofs}
\newenvironment{code}{\verbatim}{\endverbatim}
\newcommand{\myURL}{\texttt{http://study-func-prog.blogspot.com/}}
\setlength{\topmargin}{-0.4in}%
\setlength{\oddsidemargin}{0pt}%
\setlength{\evensidemargin}{0pt}%
\setlength{\textheight}{45\baselineskip}%
\setlength{\textwidth}{47zw}%
\begin{document}
\section*{Coq チートシート}
  \begin{center}
  \begin{tabular}{c||l|l||l||l|l}
  記号 & \multicolumn{2}{c||}{使用前} & $\Longrightarrow$ & \multicolumn{2}{|c}{使用後} \\
   & 仮定 & ゴール & tactic &  仮定 & ゴール \\  \hline \hline
  仮定 & \verb|p:P| & \verb|P| & \verb|exact p.| & & 証明終わり \\ \hline \hline
  $\rightarrow$ & & \verb|P -> Q| & \verb|intro p0.| & \verb|p0:P| & \verb|Q| \\ \cline{2-6}
  \verb|->| & \verb|pq: P -> Q| & \verb|Q| & \verb|apply pq.| & \verb|pq0: P -> Q| & \verb|P| \\ \hline \hline
  $\wedge$ & & \verb|P /\ Q| & \verb|split.| & & (1/2) \verb|P| \\
           & &               &               & & (2/2) \verb|Q| \\ \cline{2-6}
  \verb|/\| & \verb|pq: P /\ Q| & & \verb|destruct pq| & \verb|p0:P| & \\
  & & & \verb| as [p0 q0].|& \verb|q0:Q| & \\ \hline \hline
  $\vee$ & & \verb|P \/ Q| & \verb|left.| & & \verb|P| \\ \cline{4-6}
  & & & \verb|right.| & & \verb|Q| \\ \cline{2-6}
  \verb|\/| & \verb|pq:P \/ Q| & & \verb+destruct pq+ & (1/2) \verb|p0:P| & \\
  & & & \verb+ as [p0|q0].+ & (2/2) \verb|q0:Q| & \\ \hline \hline
  $\top$ & & \verb|True| & \verb|exact I.| & & 証明終わり \\ \hline \hline
  $\bot$ & \verb|H:False| & & \verb|elim H.| & & 証明終わり \\ \hline \hline
  $\neg$ & & \verb|~P| & \verb|intro p0.| & \verb|p0:P| & \verb|False| \\ \cline{2-6}
  \verb|~| & \verb|np:~P| & & \verb|elim np.| & \verb|np:~P| & \verb|P| \\ \hline \hline
  $\forall$ & & \verb|forall x:X,| & \verb|intro x0.| & \verb|x0:X| & \verb|f x0 y = ...| \\
  & & \verb|f x y = ...| & & & \\ \cline{2-6}
  \verb|forall| & \verb|H:forall x:X,| & \verb|f x0 = ...| & \verb|exact (H x0).| & & 証明終わり \\
  & \verb| f x = ...| & & & & \\ \hline \hline
  $\exists$ & \verb|x0:X| & \verb|exists x:X,| & \verb|exists x0.| & \verb|x0:X| & \verb|f x0 y = ...| \\
  & & \verb|f x y = ...| & & & \\ \cline{2-6}
  \verb|exists| & \verb|H:exists x:X,| & & \verb|destruct H| & \verb|x0:X| & \\
  & \verb| f x = ...| & & \verb| as [x0 H0].| & \verb|H0:f x0 = ...| & \\ \hline \hline
  $\beta \iota$簡約 & & 簡約したい式 & \verb|simpl.| & & 簡約された式 \\ \hline \hline
  等式 & & \verb|x = x| & \verb|reflexivity.| & & 証明終わり \\ \hline \hline
  書き換え & \verb|H:foo = bar| & \verb|f foo = g bar| & \verb|rewrite H.| & \verb|H:foo = bar| & \verb|f bar = g bar| \\ \cline{2-6}
  & \verb|H:foo = bar| & \verb|f foo = g bar| & \verb|rewrite <- H.| & \verb|H:foo = bar| & \verb|f foo = g foo| \\ \cline{2-6}
  & & \verb|f foo = g bar| & \verb|replace foo| & & (1/2) \verb|f baz = g bar| \\
  & & & \verb| with baz.| & & (2/2) \verb|baz = foo| \\ \hline \hline
  構築子 & \verb|H:C a = C b| & \verb|P| & \verb|injection H.| & \verb|H:C a = C b| & \verb|a=b -> P| \\ \cline{2-6}
  & & \verb|C a = C b| & \verb|f_equal.| & & \verb|a=b| \\ \cline{2-6}
  & \verb|H:C1 .. = C2 ..| & & \verb|discriminate H.| & & 証明終わり \\ \hline \hline
  \end{tabular}
  \end{center}
  \newpage
\begin{description}
\item[名前の自動生成] \verb|intro. destruct x.| などで自動的に名前を生成してくれるが、自分で名前を付ける方が良い習慣。
\item[intros] \verb|intros p q.| で複数の \verb|intro| を代用可能。
\item[対象] \verb|simpl, rewrite, replace| は \verb|simpl in H.| や \verb|simpl in *.| など tactic 対象を変更出来る。
\item[場合分け] \verb|destruct x.| で \verb|x| の (構築子毎の) 場合分けが出来る。似たものとして \verb|case_eq x.| もある。
\item[帰納法] \verb|induction x.| で \verb|x| の 帰納法が出来る。
\item[generalize] 仮定 \verb|x:X| がある時、\verb|generalize dependent x.| するとゴールが \verb|forall x:X, ...| になる。
\item[inversion] \verb|injection, discriminate| を纏めたような強力な tactic。仮定に対する場合分けが出来る。
\item[e系tactic] \verb|eauto, eapply, erewrite| などは定理名から引数を推測するので省略可能。
\item[assert] 証明の途中で補題を作りたい時に \verb|assert(H:hogehoge).| とすると、ゴールが \verb|hogehoge| に切り替わる。証明が終わると \verb|H:hogehoge| が仮定に追加される。
\item[remember] \verb|remember (f foo bar) as x.| とすると、仮定 \verb|Heqx:x = f foo bar| が追加され、\verb|f foo bar| が \verb|x| で置き換えられる。\verb|remember| して \verb|destruct, induction| することもあり。 
\item[info] \verb|info auto.| とすると、\verb|auto.|の実行した中身を調べられる。(Coq $8.3$ まで。)
\item[fold/unfold] \verb|Definition f := ...| で定義した定義を展開するのは \verb|unfold f| で、逆が \verb|fold f|。\verb|unfold f; fold f.| で元に戻るのではなく、いい感じに簡約されることがある。
\item[ring] \verb|Require Import Ring.| すると使える自動証明。\verb|ring.| で環の等式 (加減算と乗算に関する等式) を解く。 
\item[omega] \verb|Require Import Omega.| すると使える自動証明。\verb|omega.| で一次式に関する等式不等式 (正確にはPresburger算術の式) を解く。 
\end{description}
\end{document}
